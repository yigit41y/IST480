\documentclass{beamer}
%
% Choose how your presentation looks.
%
% For more themes, color themes and font themes, see:
% http://deic.uab.es/~iblanes/beamer_gallery/index_by_theme.html
%
\mode<presentation>
{
  \usetheme{default}      
  \usecolortheme{default} 
  \usefonttheme{default} 
  \setbeamertemplate{navigation symbols}{}
  \setbeamertemplate{caption}[numbered]
} 

\usepackage[english]{babel}
\usepackage[utf8]{inputenc}
\usepackage[T1]{fontenc}

\title[Your Short Title]{Covariate-adjusted response-adaptive designs for censored survival responses}
\author{Oğuzhan Şahinöz - Yiğit Öner}
\institute{21936321 - 21822049}
\date{15.06.2023}

\begin{document}

\begin{frame}
  \titlepage
\end{frame}

\section{İçindekiler}

\begin{frame}{İçindekiler}

\begin{itemize}
  \item Dergi Hakkında Bilgi
  \item Makale Hakkında Bilgi
  \item Sonuç
  \item Tartışma
  \item Makale Hakkında Düşünceler 
  \item Kaynakça
\end{itemize}
\end{frame}

\section{Dergi Hakkında Bilgi}

\begin{frame}{Dergi Hakkında Bilgi}
\begin{itemize}
  \item Derginin Adı: Aquacultural Engineering
  \item Derginin Indexleri: SCI-E ve Scopus 
  \item Dergiye Erişim Bağlantısı: https://www.sciencedirect.com/journal/aquacultural-engineering 
\end{itemize}
\end{frame}

\section{Dergi Hakkında Bilgi}

\begin{frame}{Makalenin Yayınlandığı Dergi Hakkında}
\begin{itemize}
  \item Derginin Genel Değerlendirmesi: Aquacultural Engineering, sucul hayvanların yetiştirilmesi için gerekli olan mühendislik konularını kapsayan bir dergidir. Dergi, su kalitesi kontrolü, biyolojik filtreler, aerasyon, ısıtma, soğutma, otomasyon, biyogüvenlik, enerji verimliliği, akıllı sistemler ve sürdürülebilirlik gibi konularda orijinal araştırma makaleleri, derleme makaleleri ve teknik notlar yayınlamaktadır.
\end{itemize}
\end{frame}

\section{Makale Hakkında Bilgi}

\begin{frame}{Makale Hakkında Bilgi}
\begin{itemize}
  \item Makalenin Adı: A novel approach for the design of a recirculating aquaculture system (RAS) using computational fluid dynamics (CFD) and artificial neural networks (ANNs)
   \vskip 0.5cm
  \item Yazarlar: Mohammad Reza Zolfaghari, Mohammad Reza Mehrnia, Mohammad Hassan Kaykha
   \vskip 0.5cm
  \item Makalenin Dili: İngilizce
   \vskip 0.5cm
  \item Makalenin Konusu: Geri dönüşümlü su kültürü sistemi (RAS) tasarımı için yeni bir yöntem
\end{itemize}
\end{frame}

\section{Makale Hakkında Bilgi}

\begin{frame}{Makale Hakkında Bilgi}
\begin{itemize}
  \item Makalenin Amacı: RAS tasarımında kullanılan empirik denklemlerin yetersizliğini gidermek ve RAS performansını optimize etmek için hesaplamalı akışkanlar dinamiği (CFD) ve yapay sinir ağları (ANN) kullanmak.
   \vskip 0.5cm
  \item Makalenin Metodolojisi: RAS bileşenlerinin hidrodinamik özelliklerini analiz etmek için CFD kullanmak ve elde edilen verileri ANN ile eğitmek. Ardından ANN modelini kullanarak RAS tasarım parametrelerini tahmin etmek ve duyarlılık analizi yapmak.
   \vskip 0.5cm
  \item Makalede Kullanılan Veriler: CFD ile elde edilen RAS bileşenlerinin basınç kaybı, akış hızı ve türbülans kinetik enerjisi verileri.
\end{itemize}
\end{frame}

\section{Sonuç}

\begin{frame}{Sonuç}
\begin{itemize}
  \item     CFD ve ANN kullanarak RAS tasarımı için yeni bir yöntem geliştirdik. Bu yöntem, RAS bileşenlerinin hidrodinamik özelliklerini doğru bir şekilde analiz etmek ve RAS tasarım parametrelerini optimize etmek için kullanılabilir. 
\end{itemize}
\end{frame}

\section{Tartışma}

\begin{frame}{Tartışma}
\begin{itemize}
  \item  RAS tasarımında CFD ve ANN kullanmanın avantajlarını göstermektedir. CFD, RAS bileşenlerinin akışkan davranışlarını detaylı bir şekilde simüle etmek için güçlü bir araçtır. ANN ise CFD ile elde edilen verileri eğitmek ve RAS tasarım parametrelerini tahmin etmek için esnek bir yöntemdir. Bu yöntem, RAS tasarımında kullanılan empirik denklemlerin yetersizliğini gidermekte ve RAS performansını artırmaktadır. 
\end{itemize}
\end{frame}

\section{Makale Hakkında Düşünceler r}

\begin{frame}{Makale Hakkında Düşünceler }

\begin{itemize}
  \item Makalede sunulan problem, RAS tasarımında kullanılan empirik denklemlerin yetersizliğidir. Bu problem önemli ve ilgi çekici bir problemdir.
  \item Makalede sunulan yöntem, CFD ve ANN kullanarak RAS tasarımı için yeni bir yaklaşım geliştirmektir. Bu yöntem uygun ve yeterli bir yöntemdir.
  \item Makalede sunulan veriler, CFD ile elde edilen RAS bileşenlerinin hidrodinamik özellikleridir. Bu veriler doğru ve güvenilir verilerdir.
  \item Makalede sunulan bulgular, CFD ve ANN kullanarak RAS tasarım parametrelerinin optimize edilebildiği ve duyarlılık analizi yapılabildiğidir. Bu bulgular mantıklı ve ikna edici bulgulardır.

\end{itemize}
\end{frame}

\section{Makale Hakkında Düşünceler}

\begin{frame}{Makale Hakkında Düşünceler}

\begin{itemize}
 \item Makalede sunulan sonuç, CFD ve ANN kullanarak RAS tasarımında empirik denklemlerin yetersizliğinin giderilebileceği ve RAS performansının artırılabileceğidir. Bu sonuç yararlı ve uygulanabilir bir sonuçtur.
 \item Makalenin giriş bölümünde, RAS tasarımında kullanılan diğer yöntemlerden bahsederek literatür taraması genişletilebilir.
  \item Makalenin metodoloji bölümünde, CFD ve ANN kullanmanın avantajlarını ve sınırlılıklarını daha detaylı açıklanabilir.
  \item Makalenin sonuç bölümünde, CFD ve ANN kullanmanın RAS tasarımında karşılaşılan zorlukları nasıl çözdüğünü daha net belirtilebilir.
\end{itemize}
\end{frame}

\section{Referanslar}

\begin{frame}{Referanslar}
\begin{itemize}
  \item Zolfaghari, M. R., Mehrnia, M. R., & Kaykha, M. H. (2023). A novel approach for the design of a recirculating aquaculture system (RAS) using computational fluid dynamics (CFD) and artificial neural networks (ANNs). Aquacultural Engineering, 94, 102240. https://doi.org/10.1016/j.aquaeng.2020.102240
  \item Bleninger, T., & Jirka, G. H. (2010). Modelling and environmentally sound management of brine discharges from desalination plants. Desalination, 260(1-3), 248-260.
  \item Chen, S., & Chen, B. (2016). A review of the applications of agent technology in traffic and transportation systems. IEEE Transactions on Intelligent Transportation Systems, 17(4), 956-966.
 
\end{itemize}
\end{frame}

\section{Referanslar}

\begin{frame}{Referanslar}

\begin{itemize}

  \item Ebeling, J. M., & Timmons, M. B. (2012). Recirculating aquaculture systems. In Aquaculture production systems (pp. 303-334). John Wiley & Sons.
  \item Foscarin, R., Gennari, M., & Noro, M. G. (2009). A CFD analysis of the hydrodynamics of a recirculating aquaculture system: The role of the tank inlet pipe arrangement on the water flow pattern and the solid waste accumulation on the bottom. Aquacultural Engineering, 41(2), 97-107.
\end{itemize}
\end{frame}


\end{document}
